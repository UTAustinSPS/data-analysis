\documentclass{article}
\usepackage[latin1]{inputenc}
\usepackage{enumerate}
\usepackage{hyperref}
\usepackage{graphics}
\usepackage{graphicx}
\usepackage{caption}
\usepackage{subcaption}
\usepackage{tabularx}
\usepackage{amsmath}
\newcommand{\ket}[1]{\ensuremath{\left|#1\right\rangle}}
\newcommand{\bra}[1]{\ensuremath{\left\langle#1\right|}}
\newcommand{\braket}[2]{\ensuremath{\left\langle #1 \middle| #2 \right\rangle}}
\newcommand{\obar}[1]{\ensuremath{\overline{ #1 }}}

\newcommand{\M}[0]{\emph{Mathematica}~}
% enumerate is numbered \begin{enumerate}[(I)] is cap roman in parens
% itemize is bulleted \begin{itemize}
% subfigures:
% \begin{subfigure}[b]{0.5\textwidth} \includegraphics{asdf.jpg} \caption{} \label{subfig:asdf} \end{subfigure}
\hypersetup{colorlinks=true, urlcolor=blue, linkcolor=blue, citecolor=red}
\graphicspath{ {C:/Users/Evan/Desktop/} }
\title{Assignment 5: \\ Introduction to \emph{Mathematica}\\
\large \emph{Introduction to Data Analysis for Physics}}
\author{Evan Ott and Will Beason}
\date{Spring 2014}
\setcounter{secnumdepth}{0}
\usepackage[parfill]{parskip}
\begin{document}
\maketitle
\section{Submission Requirements}
Submit the assignment to \href{mailto:data.analysis.physics@gmail.com}{data.analysis.physics@gmail.com} by Wednesday at 5pm. Just submit the \M
document you create (typically a .nb file).

\section{Problem 1}
We're going to once again use the data for Halley's comet, this time focusing on using it to learn about error propagation. Reproduced below are the
equations for mean, variance, and error propagation for independent variables (for context, review the textbook):

\begin{align*}
\langle x\rangle&=\frac{1}{N}\sum_{i=1}^Nx(i)\\
\sigma_x^2&=\frac{1}{N-1}\sum_{i=1}^N(x(i)-\langle x\rangle)^2\\
f(x_1,~x_2,~...,~x_M)&=\cdots\\
\langle f\rangle&=\frac{1}{N}\sum_{i=1}^Nf(x_1(i),~x_2(i),~...,~x_M(i))\\
\sigma_f^2&=\sum_{j=1}^M\left(\frac{\partial f}{\partial x_j}\middle|_{\langle x_1\rangle,~\langle x_2\rangle,~...,~\langle x_M\rangle}\right)^2\sigma_{x_j}^2
\end{align*}

\subsection{a}
To practice this, with the data for Halley's comet in polar coordinates $(\phi,~r)$ as before (with the \texttt{phi}$\rangle$\texttt{Mod[phi+2Pi, Pi]} correction), let's calculate
the mean observed eccentricity (which should be the same value as what you got in the last homework, or very nearly the same) and standard deviation of eccentricity. This time,
we'll assume that the semi-major axis is $a=2.662\times10^{12}m$. This gives:
$$r=\frac{a(1-\epsilon^2)}{1+\epsilon\cos(\phi)}$$
or, solving for $\epsilon$ and taking the positive root of the quadratic equation (verify this):
$$\epsilon(r,~\phi)=\frac{-r\cos(\phi)+\sqrt{r^2\cos^2(\phi)-4a(r-a)}}{2a}$$
From this and the equations above, calculate $\langle\epsilon\rangle$ and $\sigma_\epsilon$. Are these values consistent with what you got from the curve-fitting assignment?

\subsection{b}
For a function $f$ over variables $x,~y,~...$, we defined $\langle f\rangle=\sum_if(x_i, y_i, ...)/N$, which is to say the mean of values of $f$
computed from the different selections of observed quantities that
we have in our dataset. But, what if we instead calculated $f(\langle x\rangle,~\langle y\rangle,~...)$? For the eccentricity calculation above, compute this quantity
({\emph{i.e.,}} $\epsilon(\langle r\rangle, \langle\phi\rangle)$). Is it the same as $\langle f\rangle$? If not, how many standard deviations away is it? Can you think of a reason
why these two quantities might not be the same? (hint: what if we had a variable that only had values -100 and 100 [nothing in-between]?)

\subsection{c}
In the last homework, we had {\emph{Mathematica}} calculate the eccentricity $\epsilon$ and length-scale $c$ of the orbit. From this, we calculated
the semi-major axis using $a=c/(1-\epsilon^2)$. Still assuming $a=2.662\times10^{12}m$, calculate the mean and standard deviation of length-scale $c$ based on the mean and standard
deviation of eccentricity you calculated above.

\section{Problem 2}
The last {\emph{Mathematica}} problem for a few weeks!

For this one, we're just going to try a couple examples for using \texttt{ErrorListPlot}. First, for a function $f(n)=n^2+n-3$, let the standard deviation be constant $\sigma_f=3$. Plot points
for this distribution for $n\in[-10,10]$. Then, try another plot where we have $g(t)=\sin(t-.15)$ and varying $\sigma_g(t)=.7\sqrt{|\sin{t}|}$ for many points in the range $0<t<\pi$ (more
than 50). If you allow the visible range on the graph to be $-5<t<8$ and $-0.5<g<2$, you may find a shape similar to that of Starfleet from Star Trek (\url{http://i.stack.imgur.com/mVOSg.gif}).
I promise I happened to compute a similar shape and fixed it up slightly.

For this plot, remember that it's different from most. It takes
\begin{verbatim}
ErrorListPlot[{{{x1,y1}, ErrorBar[e1]}, {{x2,y2}, ErrorBar[e2]}, ... }
\end{verbatim}

In future assignments, we may take advantage of the fact that the \texttt{ErrorBar} function allows us to create error bars in both the horizontal and vertical axes, with
sizes that can differ for each of the cardinal directions.
\end{document}