%Hey class! If this document is not typesetting correctly, immediately email me
%at willbeason@gmail.com.

\documentclass[a4paper,10pt]{article}
\usepackage[margin=1in]{geometry}
\usepackage{amssymb}
\usepackage{mathtools}

\renewcommand\thesection{Problem \arabic{section} of 3:}
\renewcommand{\labelitemi}{\small$\square$}

\title{\vspace{-1in} Assignment 7: Mathematics in \LaTeX}
\author{Will Beason and Evan Ott}
\date{Due: 5pm on 5 March 2014}

\begin{document}

\maketitle

There are many useful features \LaTeX provides for typesetting mathematics. The main idea behind this assignment is to give you a small taste of what is possible. You can look for equations online, but everyone should typeset different equations. For each, create a section containing the features listed. Don't forget to load \textbf{amsmath} and \textbf{amsthm}. Try to make the mathematics somewhat consistent within each problem.


\section{Equations}
Create equations containing:
\begin{itemize}
\item a use of the equation environment,
\item a use of the align environment with a minimum of five lines of equations aligned with \&,
\item a use of the align environment with two columns
\item a use of the cases environment,
\item a use of the multiline environment spanning three lines,
\item a limit with a subscript,
\item a summation with a subscript and superscript,
\item a product with multiple conditions stacked underneath,
\item matrix multiplication,
\item an operator you define in the preamble,
\item a use of \textbackslash left and \textbackslash right around a fraction, and
\item a use of \textbackslash middle.
\end{itemize}

\section{Inline Mathematics}

Create a paragraph which:
\begin{itemize}
\item describes the mathematics going on within it,
\item has all equations inline,
\item has an integral and a derivative,
\item refers to an equation from the previous problem,
\item has a small matrix, and
\item has a small fraction.
\end{itemize}

\section{Proofs}
Create a proof containing the following features:
\begin{itemize}
\item a lemma or a theorem,
\item a proof,
\item some form of set notation, and
\item the symbol for integers or natural numbers.
\end{itemize}


\end{document}














