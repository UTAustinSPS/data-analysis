%Hey class! If this document is not typesetting correctly, immediately email me
%at willbeason@gmail.com.

\documentclass[a4paper,10pt]{article}
\usepackage[margin=0.9in]{geometry}
\usepackage{hyperref}
\usepackage{multirow}

\renewcommand\thesection{Problem \arabic{section} of 1}

\title{\vspace{-1in} Assignment 6:Introduction to
\LaTeX\\\textit{\large{Documents and Settings}}}
\author{Will Beason and Evan Ott}
\date{Due: 27 February 2014}

\begin{document}

\maketitle

\noindent\large{\textbf{Submission Requirements}}\normalsize

Submit the assignment to
\href{mailto:data.analysis.physics@gmail.com}{data.analysis.physics@gmail.com}
by Wednesday at
5pm. Just submit the \LaTeX\ document you create (typically a .tex
file). Feel free to send questions to
\href{mailto:willbeason@gmail.com}{willbeason@gmail.com} -- especially if you
want to do something cool with your assignment.

\section{}


Create a document which meets the following description. I want everyone to have
a different document, so experiment with possible settings!

\begin{enumerate}

\item Define the document class and use settings. The list below is not
complete: try Googling for more or testing ones to see what happens. Recall that
some settings are default, such as orienting the page portrait, and do not need
to be declared. (Unless you're using document types that default to landscape,
but these are rare.) Some like draft are binary and doesn't have an alternate
setting: to indicate the document is not a draft, don't include ``draft'' when
declaring the document class.

\vspace{1em}

\begin{tabular}{l||ccc|}
\textbf{Setting Description}&\multicolumn{3}{c}{\textbf{Examples}}\\ \hline
Font Size&10pt&11pt&12pt\\\hline
Paper Type&a4paper&letterpaper&a5paper\\\hline
Title Page&titlepage&notitlepage&\\\hline
Columns&onecolumn&twocolumn&\\\hline
Page Sides&oneside&twoside&\\\hline
Orientation&portrait&landscape&\\\hline
Draft&draft&&\\\hline
\end{tabular}

\vspace{1em}

\item Use the \texttt{geometry} package. The below table has some settings to
play around with. Some options take measurements. \LaTeX\ accepts many different
units, but the most common are \texttt{pt}, \texttt{in}, \texttt{cm},
\texttt{mm}. You don't need to use every option to load the package; in fact,
none are necessary. Some settings modify multiple others - \texttt{twoside}
automatically makes the inside margins larger.

\vspace{1em}

\begin{tabular}{p{1.7in}||p{1.8in}|p{1.8in}|}
\textbf{Option(s)}&\textbf{Explanation}&\textbf{Sample Settings}\\ \hline
\texttt{margin,hmargin,vmargin,}&Set all, horizontal, vertical,
&\texttt{margin=1in}\\
\texttt{lmargin,rmargin,}&left, right, top, or
bottom&\texttt{hmargin=1.5cm,vmargin=2cm}\\
\texttt{tmargin,bmargin}&margin sizes.&\texttt{lmargin=1.5in,rmargin=1in}
\\\hline
\texttt{twoside}&For a two-sided document, flips&\\
&margins on even pages.&\texttt{twoside}\\ \hline
\texttt{textwidth,textheight}&How wide the text area
is.&\texttt{textwidth=5.5in}\\\hline
\end{tabular}

\vspace{1em}

A full list of settings and more complete explanations can be found
\href{ftp://ftp.tex.ac.uk/tex-archive/macros/latex/contrib/geometry/geometry.pdf}{in
the documentation}.

\vspace{1em}

\item Include text content. This can be composed yourself or copied from
Wikipedia (or anything in the public domain, just include a link in the document
to your source). Divide the text into sections. Use at least one section,
subsection, subsubsection, paragraph, and subparagraph. Try playing around with horizontal and vertical spacing.

\item End the document.

\end{enumerate}

\noindent\large{\textbf{Hints}}\normalsize

Remember you can always view the source of this document to get ideas. Every
part of this assignment is in some way expressed in this document. Remember you
can just push Ctrl+T to quickly typeset and enter side-by-side mode. For
awesomepoints, use the \texttt{book} document class and include parts and
chapters.

\end{document}














