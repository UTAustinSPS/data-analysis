\documentclass{article}
\usepackage[latin1]{inputenc}
\usepackage{enumerate}
\usepackage{hyperref}
\usepackage{graphics}
\usepackage{graphicx}
\usepackage{caption}
\usepackage{subcaption}
\usepackage{tabularx}
\usepackage{amsmath}
\newcommand{\ket}[1]{\ensuremath{\left|#1\right\rangle}}
\newcommand{\bra}[1]{\ensuremath{\left\langle#1\right|}}
\newcommand{\braket}[2]{\ensuremath{\left\langle #1 \middle| #2 \right\rangle}}
\newcommand{\obar}[1]{\ensuremath{\overline{ #1 }}}
% enumerate is numbered \begin{enumerate}[(I)] is cap roman in parens
% itemize is bulleted \begin{itemize}
% subfigures:
% \begin{subfigure}[b]{0.5\textwidth} \includegraphics{asdf.jpg} \caption{} \label{subfig:asdf} \end{subfigure}
\hypersetup{colorlinks=true, urlcolor=blue, linkcolor=blue, citecolor=red}
\graphicspath{ {C:/Users/Evan/Desktop/} }
\title{Assignment 1: \\ Introduction to \emph{Mathematica}\\
\large \emph{Introduction to Data Analysis for Physics}}
\author{Evan Ott and Will Beason}
\date{Spring 2014}
\setcounter{secnumdepth}{0}
\usepackage[parfill]{parskip}
\begin{document}
\maketitle
\section{Submission Requirements}
This week, submit the assignment to \href{mailto:data.analysis.physics@gmail.com}{data.analysis.physics@gmail.com} by Wednesday at 5pm. Doing so will let
us review the assignment ahead of time so we can appropriately address questions and challenges during class. Turning in homework on time (even if incomplete) is absolutely
preferred to not turning it in until later. This will be the format each week. You may, of course,
turn it in late at no penalty, but you are \emph{strongly} encouraged to complete these assignments as they are given as to not fall behind. We will cover a broad
range of topics, many of which will build on each other.

This week, just submit the \emph{Mathematica} document you create (typically a .nb file).

As you find practice problems in the book, you are encouraged to try them out (some will take a few moments, others more time), but they are not assigned. We'll
be happy to take a look at them, answer questions, etc. however, whenever you choose!
\section{Problem 1}
First, let's create some functions! Let $f(x)=1-x^2/2$. Then, create $f(g(x))=x$ (you'll need to do a little math - just a little - be sure to show that the function is correct).
Then run the $g$ function on a list \begin{verbatim}myList=(1, 2, 4, 8, 5, 7)\end{verbatim}, then take the square of the norm. What is the answer?
\section{Problem 2}
Use the following code, then write the correct sequence of indexing and mathematical operations to produce the output below.

Code:
\begin{verbatim}
myList2 = {0, 1, 4, 9 , 16, 25, 36, 49}
\end{verbatim}

Output:
\begin{verbatim}
{1, 9, 36}
{0, 1, 4, 9}
{1, 4, 9, 16, 25, 36, 49, 64}
\end{verbatim}
\section{Problem 3}
Do something interesting! Take some time to explore the few topics discussed this week. Do something you find interesting and show it to us!

\end{document}